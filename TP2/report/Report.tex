\documentclass[a4paper]{report}
\usepackage[utf8]{inputenc}
\usepackage[T1]{fontenc}
\usepackage{a4wide}
\usepackage{hyperref}
\hypersetup{pdftitle={Processing an Angolan Newspaper},
pdfauthor={Pedro Mendes},
colorlinks=true,
urlcolor=blue,
linkcolor=black}
\usepackage{subcaption}
\usepackage[cache=false]{minted}
\usepackage{listings}
\usepackage{booktabs}
\usepackage{multirow}
\usepackage{appendix}
\usepackage{tikz}
\usetikzlibrary{positioning,automata,decorations.markings}

\begin{document}

\title{Processing the CETEMPublico's corpus}
\author{Pedro Mendes (a79003)}
\date{\today}

\begin{center}
    \begin{minipage}{0.75\linewidth}
        \centering
        \includegraphics[width=0.4\textwidth]{eng.jpeg}\par\vspace{1cm}
        \vspace{1.5cm}
        \href{https://www.uminho.pt/PT}
        {\color{black}{\scshape\LARGE Universidade do Minho}} \par
        \vspace{1cm}
        \href{https://www.di.uminho.pt/}
        {\color{black}{\scshape\Large Departamento de Informática}} \par
        \vspace{1.5cm}
        \maketitle
    \end{minipage}
\end{center}

\begin{abstract}
    \begin{center}
        This project's goal is to analyse the \textit{CETEMPublico}'s corpus,
        compiled from publications of the \textit{Publico}
        newspaper~\cite{CETEMPublico}. This document's purpose is to assist
        researchers in the development of software that needs to process the
        Portuguese language, as well as, anyone studying the Portuguese
        language.

        Simple examples of such processing are demonstrated in this paper.
    \end{center}
\end{abstract}

\tableofcontents

\pagebreak

\chapter{Introduction}

This project aims to parse the \textit{CETEMPublico}'s corpus to produce some
statistics using the \textit{gawk} programming language. \textit{Gawk} is a
domain specific language designed for parsing CSV and CSV-like files making it
a very competent choice for the proposed problem, it does have it's
limitations, as will be shown in the on the last chapter.

First we'll analyse the problem, see what needs to be implemented and what
challenges need to be overcome to implement said features.

Next we'll look at the solutions to the proposed problems, dedicating a
section to the solution of each problem.

\chapter{Problem}

The following analysis must be made:
\begin{itemize}
    \item Count the number of sentences, paragraphs, extracts and
        multi word expressions.
    \item Find and count the multi word expressions (MWE).
    \item Find and count the verbs.
    \item Build the implicit dictionary in the corpus.
\end{itemize}

The input file uses XML tags to annotate it's contents, therefore the gawk
scripts will have to look for these during parsing.

Multi Word Expressions span more then one line, thus the line based parsing
will have to be adapted.

As is usual with these kinds of file processing tasks, the input is millions of
lines long which means that care must be taken to not accumulate too much data
in program memory.

\chapter{Solution}

\section{Requirement 1: Counting}\label{sec:count}

The first requirement is relatively simple, the corpus contains a tag for each
of the items that need to be counted, therefore my job was simply to count
these tags.

\begin{figure}[H]
    \centering
    \RecustomVerbatimEnvironment{Verbatim}{BVerbatim}{}
    \inputminted[firstline=2]{awk}{../count.awk}
    \caption{\textbf{count.awk:} Gawk script to count sentences, paragraphs,
    extracts and MWE}\label{code:count}
\end{figure}

As demonstrated in Figure~\ref{code:count} each tag has a distinct regular
expression to capture it which maps to a simple counting action. At the end the
program simply prints the number of occurrences of each tag.

\section{Requirement 2: Multi Word Expressions}\label{sec:mwe}

The second requirement entails counting the MWEs in the file. Because each MWE
uses more than one line there needs to be a variable tracking the context the
parser is in.

\begin{figure}[H]
    \centering
    \RecustomVerbatimEnvironment{Verbatim}{BVerbatim}{}
    \inputminted[firstline=2]{awk}{../mwe.awk}
    \caption{\textbf{mwe.awk:} Gawk script to find and count MWEs}
\end{figure}

As can be seen the \texttt{mwe} variable tracks whether the parser is inside a
MWE\@. In addition to this, the \texttt{current} variable accumulates the MWE
to be stored in the \texttt{expressions} associative array for counting. The
script finishes by printing every expression along with it's occurrence count.

\section{Requirement 3: Verbs}\label{sec:verbs}

Verbs in the corpus are distinguished by the fifth column, if it's value starts
with a \texttt{V} then the associated word is a verb. Afterwords the script
stores the field in the forth column since this is the verb in it's infinitive
form, as opposed to the conjugated form that is in the first field.

\begin{figure}[H]
    \centering
    \RecustomVerbatimEnvironment{Verbatim}{BVerbatim}{}
    \inputminted[firstline=2]{awk}{../verbs.awk}
    \caption{\textbf{verbs.awk:} Gawk script to find and count all verbs in
    their infinitive form}
\end{figure}

The script finishes by printing out all the verbs and their occurrences.

\section{Requirement 4: Dictionary}\label{ssec:dictionary}

A dictionary in this context is list of words group with their lemmas. The
lemmas, on the other hand, have associated with them their respective part of
speech.

\begin{figure}[H]
    \centering
    \RecustomVerbatimEnvironment{Verbatim}{BVerbatim}{}
    \inputminted[firstline=3,lastline=6]{awk}{../dictionary_json.awk}
    \caption{Excerpt of a gawk script to filter and store valid
    words}\label{code:dictionary}
\end{figure}

The script used uses a single filter to obtain all the valid words (see
Figure~\ref{code:dictionary}). This filter will first remove lines started by
XML tags, and second filter any words that include punctuation, numbers, and
other non-letter symbols. Then in the action part of this filter the first
letter is capitalized and inserted into the dictionary associating the lemma
found (\texttt{\$1}) and the lemma's part of speech (\texttt{\$5}).

For the \texttt{END} action, there are two versions of the script: One that
outputs a JSON string and one that creates a series of html files to make
browsing the dictionary easier.

\subsection{JSON}\label{ssec:dict-json}

The JSON string output by this script groups words with their lemmas and the
lemmas with their parts of speech. It aims to be more or less direct
translation of the way they are stored in memory (see
Figure~\ref{code:dictionary}).

In the example in Figure~\ref{out:json} we can see represented an object
containing a dictionary which contains a words field with the list of words,
each one containing a list of lemmas. Each lemma with a list of parts of speech
it was found in. This is a more or less direct translation of the way the
dictionary is stored in memory.

\begin{figure}[H]
    \centering
    \RecustomVerbatimEnvironment{Verbatim}{BVerbatim}{}
    \inputminted{json}{./dictJson.tex}
    \caption{Excerpt of the output JSON string}\label{out:json}
\end{figure}

\subsection{HTML}\label{ssec:dict-html}

The second version of the script outputs the dictionary to html files with the
following organization. An \textit{index.html} file that lists all the words in
the dictionary as links to other \textit{word.html} files that list the
different lemmas and their parts of speech.

The list in the index file is ordered alphabetically and includes links at the
top to jump the different letters of the alphabet.

This implementation, however, is very slow due to the sorting used by
\textit{gawk} not being intended for use case. For example, to process the
largest file provided the script takes 40 minutes to complete. To solve this
issue a implementation of the sorting and generation of the html files was made
in \href{https://www.rust-lang.org/}{Rust} a system's programming language that
has performance comparable to that of \textit{C} and \textit{C++}. With this
the time it takes to process the largest file provided was cut to 30 seconds.

To achieve this the \textit{gawk} script opens a named pipe and then starts the
Rust program in the background, setting it up to read from the pipe, afterwords
it simply writes the filtered and valid words to the pipe. On the other side,
the Rust code is very similar to the previous, \textit{gawk} only code, keeping
the same data structure, only using a binary tree to store the words instead of
an associative array.

Both versions of the program were kept for evaluation purposes.

\begin{figure}[H]
    \centering
    \RecustomVerbatimEnvironment{Verbatim}{BVerbatim}{}
    \inputminted[firstline=2]{awk}{../dictionary_html_fast.awk}
    \caption{\textbf{dictionary\_html\_fast.awk:} Gawk script to build a
    dictionary out of the corpus, using a second program to process the
    filtered words}\label{code:count}
\end{figure}

\chapter{Listing of the produced files}

\begin{itemize}
    \item \textbf{count.awk} Explained in Section~\ref{sec:count}
    \item \textbf{mwe.awk} Explained in Section~\ref{sec:mwe}
    \item \textbf{verbs.awk} Explained in Section~\ref{sec:verbs}
    \item \textbf{dictionary\_json.awk} Explained in
        Subsection~\ref{ssec:dict-json}
    \item \textbf{dictionary\_html.awk} and \textbf{dictionary\_html\_fast.awk}
        Explained in Subsection~\ref{ssec:dict-html}
\end{itemize}

\chapter{Conclusion}

In conclusion, \textit{gawk} is a very powerful tool for filtering and
reorganizing text that gets a lot of work done in very few lines of code. On
the other hand it is not intended for complex data processing and
reorganization. Because of this it is my opinion that when using it one has to
know it's limits or risk wasting more time then needed.

The tool also fits incredibly well with the rest of the Unix system ecosystem,
reading from the standard input by default and writing to the standard output,
it can be used to process in conjunction with other common tools through the
use of pipes.

In regards to the work done in this assignment, most tasks were relatively easy
to complete due to \textit{gawk}'s powerful syntax, only ever falling short on
the aforementioned performance and complex processing, which the tool was
never meant to be used for in the first place.

\bibliographystyle{plain}
\bibliography{Report}

\end{document}

